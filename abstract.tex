% 中文摘要
\begin{abstractzh}

篇章連接詞真的好重要。\\

\noindent
關鍵字:篇章結構分析

\end{abstractzh}

% 英文摘要
\begin{abstracten}

Discourse connectives are important clues for identifying discourse
relations in Chinese texts. However, the ambiguity involved makes
it a challenge to extract true connectives. In this thesis, we investigate
three types of ambiguities together. First, a connective component may have 
discourse and non-discourse usages. Second, there may be multiple possible
linkings for paired connectives. Third, a connective may have different
meanings, signaling different discourse relations. A unified approach 
to resolve these ambiguities in unrestricted text is
proposed. The experimental results show that GloVe (Global Vectors for 
Word Representation) is the most powerful feature set for both discourse 
usage and linking disambiguation. The entire system achieves the 
macro-average F1 of xx.xx\% on Chinese Discourse Treebank. \\ 

\noindent
Keywords: Discourse Parsing, Discourse Connective
\end{abstracten}

