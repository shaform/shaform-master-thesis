\begin{figure}[ht]
\centering
    \begin{tikzpicture}
        \begin{axis}[
            width=15cm,
            bar width = 7pt,
            enlarge x limits=0.05,
            symbolic x coords={
            并,其中, 也, 而, 但, 还, 使, 以, 为, 同时,
            后,由于, 因此, 如, 又, 为了, 因为, 而且, 如果, 後,
            此外,虽然-但, 但是, 从而, 然而
            },
            ybar,
            nodes near coords={\pgfmathprintnumber[/pgf/number format/assume math mode]{\pgfplotspointmeta}},
            x tick label style={font=\small,align=center,rotate=70},
            xtick=data,
            xticklabel style={/pgf/number format/assume math mode},
            yticklabel style={/pgf/number format/assume math mode},
          ]
            \addplot[ybar,fill=blue] coordinates {
            (并,208)
            (其中,154)
            (也,133)
            (而,70)
            (但,69)
            (还,68)
            (使,56)
            (以,52)
            (为,47)
            (同时,46)
            (后,45)
            (由于,37)
            (因此,30)
            (如,29)
            (又,26)
            (为了,22)
            (因为,21)
            (而且,20)
            (如果,19)
            (虽然-但,16)
            (此外,16)
            (後,16)
            (但是,15)
            (然而,13)
            (从而,13)
            };
        \end{axis}
    \end{tikzpicture}
\caption{\label{i:connective-types} Top 25 classes of connectives. }
\end{figure}
