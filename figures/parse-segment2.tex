\begin{figure}[ht]
\centering
\tikzset{every tree node/.style={align=center},
    sibling distance=-1pt}
\begin{tikzpicture}
\Tree[.\node(ip2){IP};
         [.\node(ip3){IP}; \edge[roof]; {[现已建成包钢稀土一、]\ldots} ]
         [.PU , ]
         [.\node(ip1){IP}; [.\node(np1){NP}; [.NN {\underline{其中}} ] ]
             [.VP [.VE 有 ]
             [.NP [.CP [.IP [.NP [.NN 世界 ] ]
                         [.VP [.ADVP [.AD 最 ] ]
                             [.VP [.VA 大 ] ] ] ]
                     [.DEC 的 ] ]
                  [.NP \edge[roof]; {\ldots} ] ] ] ] ]

\draw [->,color=blue] (np1) [bend left] to (ip1);
\draw [->,color=blue] (ip1) [bend right] to (ip2);
\draw [->,color=blue] (ip2) [in=-180,out=180] to (ip3);

\end{tikzpicture}
\caption{\label{i:parse-segment2} A sub- parsing tree for the segments
with connective 其中. }
\end{figure}
