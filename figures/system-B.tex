\begin{figure}[ht]
\centering
\tikzstyle{block} = [rectangle, draw, node distance=1.5cm,
    text width=10em, text centered, rounded corners, minimum height=3em]
\tikzstyle{line} = [draw, -latex']
\tikzstyle{cloud} = [draw, ellipse, node distance=1.5cm,
    minimum height=2em]
\tikzstyle{container} = [draw, rectangle, dashed, inner sep=0.8em]

\begin{tikzpicture}[node distance = 2cm, auto]
    % Place nodes
    \node [cloud] (components) {component candidates};

    \path (components)+(-2.5,-2)
        node [block] (eliminate1) {B1. Eliminate non-discourse components};

    \node [block, right=3em of eliminate1] (generate2) {B2. Generate all connective candidates};

    \node [block, below of=generate2] (eliminate2) {B2. Eliminate unlikely connective candidates};

    \node [block, below of=eliminate1] (generate1) {B1. Generate all connective candidates};

    \path (generate1)+(2.5,-2) node [block] (linking) {C. Resolve linking ambiguities};

    \node [cloud, below of=linking] (connectives) {connectives};

    % Draw edges
    \path [line] (eliminate1) -- (generate1);
    \path [line] (generate1) -- (linking);
    \path [line] (generate2) -- (eliminate2);
    \path [line] (eliminate2) -- (linking);
    \path [line,dashed] (linking) -- (connectives);
    \path [line,dashed] (components) -- (eliminate1);
    \path [line,dashed] (components) -- (generate2);
\end{tikzpicture}

\caption{\label{i:system-B} System overview for alternative methods
to extract connectives. } In B1, we eliminate non-discourse components first,
and generate all possible candidates by linking the components. In B2,
we do not eliminate components, but use all possible connective candidates, and
disambiguate discourse usages on the connective level. In fact, by eliminating
unlikely connective candidates, B2 actually solves some of linking ambiguities
before feeding the remaining candidates to C.
\end{figure}
