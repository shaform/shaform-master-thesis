%
%   Chapter Datasets
%
%   Yong-Siang Shih
%   R.O.C.104.07
%
\chapter{Datasets}
\label{c:datasets}

In this chapter, we discuss the datasets we use in our study, and
provide some statistics and analysis.

\section{Chinese Discourse Treebank (CDTB)}

Chinese Discourse TreeBank (CDTB)~\citep{li2014building} is a discourse
corpus that contains 500 articles from the Chinese Treebank (CTB)~\citep{xue2005penn}.
Totally, it contains 2,342 paragraphs.
Inspired by RST and PDTB, they proposed the Connective-driven Dependency Tree (CDT)
scheme to annotate the documents.

In CDTB, each paragraph is represented as a discourse tree. The leaf nodes are the
elementary discourse units (EDUs) for the text, while internal nodes are discourse
relations between discourse units. This is very similar to RST-DT.
Consider the example sentences (S~\ref{sent:cdtb}) from \cite{li2014building}.

\begin{sent}{sent:cdtb}{}
[浦东开发开放是一项振兴上海,建设现代化经济、贸易、金融中心的跨世纪
工程,]\textsuperscript{1}
[(因此) 大量出现的是以前不曾遇到过的新情况、新问题。]\textsuperscript{2}
[(对此),浦东\{不是\}简单的采取“干一段时间,等积累了经验以后再制定法规条例”的做法,]\textsuperscript{3}
[\{而是\}借鉴发达国家和深圳等特区的经验教训,]\textsuperscript{4}
[<并且>聘请国内外有关专家学者,]\textsuperscript{5}
[<并且>积极、及时地制定和推出法规性 文件,]\textsuperscript{6}
[\{使\}这些经济活动一出现就被纳入法制轨道。]\textsuperscript{7}

(
Pudong's development and opening up is a century-
spanning undertaking for vigorously promoting Shanghai and
constructing a modern economic, trade, and financial center.
Because of this, new situations and new questions that have
not been encountered before are emerging in great numbers.
In response to this, Pudong is not simply adopting an approach
of "work for a short time and then draw up laws and regulations
only after experience has been accumulated.”
Instead, Pudong is taking advantage of the lessons from experience of
developed countries and special regions such as Shenzhen,
by hiring appropriate domestic and foreign specialists and
scholars,
actively and promptly formulating and issuing
regulatory documents.
So these economic activities are incorporated into the sphere
of influence of the legal system as soon as they appear.
)
\end{sent}

The discourse structure is shown in Figure~\ref{i:cdtb-tree}. Importantly,
the set of discourse relation types is organized as a three-level hierarchy,
where the third level are the discourse connectives themselves. The relation
types are shown in Figure~\ref{i:three-level}. The top-level categories have
four relation types, while the second level have 17 types. For an implicit
relation, the connective that could be inserted is annotated. This is denoted
by <> in Figure~\ref{i:cdtb-tree}. Totally, there are 282 types of connectives.
For each connective of an explicit relation,
they also annotated whether it can be deleted as denoted by () or cannot be
deleted as denoted by \{\}.
If an appropriate connective to be inserted cannot be found, an internal
node may only have a second level relation type as well.

% i:cdtb-tree
\begin{figure}[!htbp]
\centering
\tikzset{every tree node/.style={align=center},
    level distance=40pt,
    sibling distance=6pt}
\begin{tikzpicture}
\Tree[.{concessive \\ 尽管 \ldots 但 (although \ldots but)}
       [.{continue \\ (以后) (later)}
         [.{coordination \\ 还 (still)}
           [ (a) ]
           [ (b) ]
         ]
         [ (c) ]
       ]
       [.{explanation}
         [ (d) ]
         [.{coordination}
           [ (e) ]
           [ (f) ]
           [ (g) ]
         ]
       ]
     ]

\end{tikzpicture}

\caption{\label{i:cdtb-tree} A discourse tree in CDTB. }
\end{figure}


% i:three-level
\begin{figure}[!htbp]
\begin{multicols}{2}
\begin{itemize}
  \item[] \textbf{causality}
  \begin{itemize}
    \item cause-result 
    \begin{itemize}
      \item[-] because \ldots
    \end{itemize}
    \item inference
    \begin{itemize}
      \item[-] so that \ldots
    \end{itemize}
    \item hypothetical
    \begin{itemize}
      \item[-] if \ldots
    \end{itemize}
    \item purpose
    \begin{itemize}
      \item[-] in order to \ldots
    \end{itemize}
    \item condition
    \begin{itemize}
      \item[-] only \ldots
    \end{itemize}
    \item background
    \begin{itemize}
      \item[-] background \ldots
    \end{itemize}
  \end{itemize}

  \item[] \textbf{transition}
  \begin{itemize}
    \item transition
    \begin{itemize}
      \item[-] but \ldots
    \end{itemize}
    \item concessive
    \begin{itemize}
      \item[-] although \ldots
    \end{itemize}
  \end{itemize}

  \vfill
  \columnbreak

  \item[] \textbf{coordination}
  \begin{itemize}
    \item coordination
    \begin{itemize}
      \item[-] and \ldots
    \end{itemize}
    \item continue
    \begin{itemize}
      \item[-] first \ldots second \ldots
    \end{itemize}
    \item progressive
    \begin{itemize}
      \item[-] in addition \ldots
    \end{itemize}
    \item selectional
    \begin{itemize}
      \item[-] or \ldots
    \end{itemize}
    \item inverse
    \begin{itemize}
      \item[-] compared with \ldots
    \end{itemize}
  \end{itemize}

  \item[] \textbf{explanation}
  \begin{itemize}
    \item explanation
    \begin{itemize}
      \item[-] which including \ldots
    \end{itemize}
    \item summary-elaboration
    \begin{itemize}
      \item[-] in a word \ldots
    \end{itemize}
    \item example
    \begin{itemize}
      \item[-] e.g. \ldots
    \end{itemize}
    \item evaluation
    \begin{itemize}
      \item[-] evaluation \ldots
    \end{itemize}
  \end{itemize}
\end{itemize}
\end{multicols}

\caption{\label{i:three-level} A three-level set of discourse relations in
CDTB. }
\end{figure}


During our experiments, we found some errors in the annotation and corrected
them. After the correction, there are 2,131 connective components annotated
for the explicit relations, and they form 1,813 connectives instances. The length
distribution of these connectives are shown in Table~\ref{t:connective-length}.
Most connectives only have one or two connective components.

% t:connective-length
\begin{table}[ht]
\centering
\begin{tabular}{|c|c|c|c|c|c|c|}
\hline
\bf \#Components        & 1    & 2   & 3  & 4 & 6 & 7 \\ \hline
\bf \#Connective Types  & 143  & 108 & 15 & 6 & 1 & 1 \\ \hline
\bf \#Instances         & 1544 & 235 & 24 & 8 & 1 & 1 \\ \hline
\end{tabular}
\caption{\label{t:connective-length} Lengths of connectives.}
This tables shows how many connective types have a specific length
and how many instances they totally appear. The number of types
is counted by the surface forms for the connectives.
\end{table}


Many of the connective types can be used for different 1-level and 2-level
relation types. As shown in Table~\ref{t:connective-type}, while most connective
types are only used for 1 relation types, the ambiguous connectives have many
instances.

% t:connective-type
\begin{table}[ht]
\centering
\begin{tabular}{|c|c|c|c|c|c|}
\hline
\#Top-level Relation Types & 1    & 2   & 3   & 4   & 5  \\ \hline
\#Connective Classes       & 258  & 15  & 1   & -   & -  \\ \hline
\#Instances                & 1388 & 355 & 70  & -   & -  \\

\hhline{|=|=|=|=|=|=|}

\#2nd-level Relation Types & 1    & 2   & 3   & 4   & 5  \\ \hline
\#Connective Classes       & 243  & 24  & 5   & 1   & 1  \\ \hline
\#Instances                & 1030 & 379 & 126 & 208 & 70 \\ \hline

\end{tabular}
\caption{\label{t:connective-type} Relation ambiguity of connectives. }
\end{table}


\section{NTU PN-Gram Corpus}

NTU PN-Gram Corpus~\citep{yu2012development} is a large-scale corpus that was
constructed by POS-tagging the Chinese texts from the ClueWeb09
dataset~\citep{callan2009clueweb09}. In NTU PN-Gram, there are totally 173,741,587
web pages containing 141,179,769,123 tokens. In our study, we use a subset that
was extracted by \cite{huang2014interpretation}. They extracted sentences that
meet three conditions:

\begin{enumerate}
\item Each sentence must have exactly two clauses.
\item Each sentence must have exactly one instance of connective.
\item Each of the two clauses must not have more than 20 Chinese characters.
\end{enumerate}

We use an intermediate result that contains 43,010,050 sentences and eliminated
the duplicate sentences. 21,217,147 sentences were left, containing
326,996,602 tokens.

We use the resulting sentences to train word embeddings for further use.
We have created 400-dimensional embeddings by GloVe tool~\citep{pennington2014glove}
and word2vec tool~\citep{mikolov2013efficient,mikolov2013distributed} with CBOW and
SKIP-GRAM models. These vectors are used as features as discussed in Chapter~\ref{c:method}.
