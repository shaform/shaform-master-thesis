%
%   Chapter Related Works
%
%   Yong-Siang Shih
%   R.O.C.104.07
%
\chapter{Related Work}
\label{c:related}

This section is organized as follows.

\section{English Discourse Corpus}

Two popular English large-scale discourse corpora are available for researchers.

In the Rhetorical Structure Theory Discourse Treebank (RST-DT)~\citep{Carlson01building},
there are 385 \textit{Wall Street Journal (WSJ)} articles selected from
the Penn Treebank~\citep{marcus1993building}. These articiles were annotated under
the \textit{Rhetorical Structure Theory (RST)}~\citep{mann-thompson88}.
Each article was segmented into \textit{elementary discourse units (EDUs)}, and the
discourse relations between these EDUs were annotated in a tree structure manner.
Consider the example sentences (S~\ref{sent:rst})  from \cite{Carlson01building}.

\begin{sent}{sent:rst}{}
[Still, analysts don't expect the buy-back to significantly affect per-share earnings in the short
term.]\textsuperscript{16} [``The impact won't be that great,'']\textsuperscript{17}
[said Graeme Lid gerwood of First Boston Corp.]\textsuperscript{18}
[This is in part because of the effect]\textsuperscript{19}
[of having to average the number of shares outstanding,]\textsuperscript{21}
[she said.]\textsuperscript{20} [In addition,]\textsuperscript{22}
[Mrs. Lidgerwood said,]\textsuperscript{23}
[Norfolk is likely to draw down its cash initially]\textsuperscript{24}
[to finance the purchases]\textsuperscript{25}
[and thus forfeit some interest income.]\textsuperscript{26}
\end{sent}

The discourse annotations are shown in Figure~\ref{i:rst-tree}. Each relation is represented
by an internal node, while each EDU is represented by a leaf node. A discourse relation is
annotated between different groups of EDUs. For example, there is a \textit{elaboration-additional}
relation between (17-21) and (22-26). The hierarchy annotations enable the analysis of
the complex discourse structure.

% i:rst-tree
\begin{figure}[ht]
\centering
\begin{tikzpicture}
\Tree[.relation~1
       [.relation~2
         [ (a) ]
         [ (b) ]
       ]
       [.relation~3
         [ (c) ]
         [ (d) ]
       ]
     ]

\end{tikzpicture}

\caption{\label{i:rst-tree} A discourse sub-tree. }
\end{figure}


Comparatively, the Penn Discourse Treebank 2.0 (PDTB2)~\citep{Prasad08thepenn}
is a much larger corpus. The annotations were done on 2159 articles from
the WSJ corpus of the Penn Treebank~\citep{marcus1993building}. They adopted a lexically-grounded
approach, annotating explicit relations signalled by connectives without constructing higher-level
structures as done in RST-DT. Therefore, the arguments do not span over long ranges. In fact,
more than 90\% of the arguments appear in the same sentence or the exact immediately
preceding sentence as pointed out by \cite{kong2014a}. In addition, they annotated implicit relations
for successive pairs of sentences.


\section{English Discourse Research}

Many groups have investigated different subtasks of English
discourse parsing on this corpus. \cite{pitler2009using} trained a maximum
entropy classifier with various syntactic features to identify explicit discourse
connectives in PDTB2 and achieved the F1 of 94.19\%. Additionally, they used
a Naive Bayes classifier to classify the relation type of each connective and
archived the accuracy of 94.15\%. Such high performance is already comparable to human annotator agreement.



\section{Chinese Discourse Corpus}

Early days research

HIT-CDTB
CDTB
%DCTD

\section{Chinese Discourse Research}



fully explain several research
%In the past, researchers investigated English discourse connective identification
%on Penn Discourse Treebank~\cite{Prasad08thepenn}. Pitler and Nenkova~\shortcite{pitler2009using} 
%used syntactic features to deal with discourse usage and
%relation type ambiguities. Various features were
%experimented to improve the results~\cite{faiz2013identifying,j2013disambig}.
%Ghosh~\shortcite{ghosh2012end} and Lin et al.~\shortcite{lin2014pdtb}
%attempted to build an end-to-end discourse parser.
%
%T'sou et al.~\shortcite{t1998automatic,t1999applying,t2000enhancement} and Chan
%et al.~\shortcite{chan2000mining} investigated connective detection in Chinese texts
%as a part of a tagging system. Hu et al.~\shortcite{hu2009research}
%developed a fully automatic system to extract connective components.
%They removed words commonly used in non-discourse contexts from the dictionary
%and improved accuracy from 0.5866 to 0.8988.
%Hu et al.~\shortcite{hu2011research} dealt with linking ambiguity.
%The above work focused on relatively short sentence groups.
%Zhou~\shortcite{zhou2012cross} and Li, Carpuat et al.~\shortcite{li2014cross} employed
%cross-lingual information to deal with discourse usage and relation type ambiguity.
%Their reported F scores for discourse usage identification are 0.7933 and 0.7163, respectively.
%Li, Carpuat et al.~\shortcite{li2014cross} used 5-way classification to evaluate discourse usage
%and relation type classification together and obtained overall accuracy of 0.7020.

