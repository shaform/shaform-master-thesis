%
%   Chapter Related Works
%
%   Yong-Siang Shih
%   R.O.C.104.07
%
\chapter{Related Work}
\label{c:related}

In this chapter, we discuss some related work for English and Chinese discourse
analysis. We firstly introduce some English discourse corpora and the
related discourse researches. Afterwards, we continue with the Chinese
discourse corpora and the related Chinese researches.

\section{English Discourse Corpora}

In this section, we introduce two popular English large-scale discourse
corpora that are available for researchers.

In the Rhetorical Structure Theory Discourse Treebank (RST-DT)~\citep{Carlson01building},
there are 385 \textit{Wall Street Journal (WSJ)} articles selected from
the Penn Treebank~\citep{marcus1993building}. These articles were annotated under
the \textit{Rhetorical Structure Theory (RST)}~\citep{mann-thompson88}.
Each article was segmented into \textit{elementary discourse units (EDUs)}, and the
\textit{rhetorical relations} between these EDUs were annotated in a tree structure
manner.

Each relation is represented by an internal node, while each EDU is represented
by a leaf node. A discourse relation is annotated between different spans of EDUs.
This is represented by the children of an internal node. Each relation can
either be binary or multi-child. Figure~\ref{i:rst-tree} shows an example sub-tree.
Relation 1 has two arguments (a-b) and (c-d), and relation 2 has two
arguments (a) and (b).
The hierarchical annotations enable the analysis of the complex discourse structure.

%i:rst-tree
\begin{figure}[!htbp]
\centering
\begin{tikzpicture}
\Tree[.relation~1
       [.relation~2
         [ (a) ]
         [ (b) ]
       ]
       [.relation~3
         [ (c) ]
         [ (d) ]
       ]
     ]

\end{tikzpicture}

\caption{\label{i:rst-tree} A discourse sub-tree. }
\end{figure}


Comparatively, the Penn Discourse Treebank 2.0 (PDTB2)~\citep{Prasad08thepenn}
is a much larger corpus. The annotations were done on 2,159 articles from
the WSJ corpus of the Penn Treebank~\citep{marcus1993building}. They adopted a lexically-grounded
approach, annotating explicit relations signalled by connectives without constructing higher-level
structures as done in RST-DT. Therefore, the arguments do not span over long ranges. In fact,
more than 90\% of the arguments appear in the same sentence of the connective or the
exact immediately preceding sentence as pointed out by \cite{kong2014a}.
In addition, they annotated implicit relations for successive pairs of sentences.


\section{English Discourse Researches}

Many groups have investigated different subtasks of English
discourse parsing on PDTB2. \cite{pitler2009using} trained a Maximum
Entropy classifier with various syntactic features to identify explicit discourse
connectives and achieved the F1 of 94.19\%. Additionally, they used
a Naive Bayes classifier to classify the relation type of each connective and
achieved the accuracy of 94.15\%. Such high performance was already comparable to human
annotator agreement. They have found that the degree of relation type ambiguity
for connectives is low in English. In fact, an accuracy of 93.67\% can be achieved
by using the strings of connectives as the only features. \cite{wellner2009sequence}
also experimented with constituency-based and dependency-based features to identify
explicit connectives with Logistic Regression classifier. In addition, they proposed
a sequential ranking model to jointly identify discourse connectives and their arguments.
They developed a dependency-based discourse parsing system. \cite{faiz2013identifying} combined
various feature sets to improve the results on connective identification. \cite{j2013disambig} pointed
out that while a few classes of connectives constitute most of connective instances, most articles contain
at least one low-frequency connective. Therefore, macro-averaged scores over different connective
classes may be a better metric. They also showed that simple lexical features such as POS tags perform
well in this experimental setting when golden standard parsing tree is not available.

\cite{dines2005attribution} used a tree subtraction algorithm to extract arguments for
intra-sentential subordinating conjunctions. \cite{wellner2007auto} considered the task of identifying
arguments for discourse connectives. They used a log-linear ranking model to extract the heads of arguments.
\cite{elwell2008discourse} followed the work to identify the heads and showed that
interpolating connective specific models with general models can improve performance.
\cite{ghosh2011shallow,ghosh2012global} formulated the problem as sequence labeling task
and used Conditional Random Fields (CRFs) to tackle the problem.
\cite{kong2014a} developed a constituent-based approach to extract arguments.
\cite{lin2014pdtb} build an end-to-end discourse parser for PDTB2.


As RST-DT provides hierarchical discourse structure annotations. There are also many
attempts to construct these discourse structures automatically. \cite{soricut2003sentence}
investigated sentence-level discourse parsing. They built discourse segmenter to
segment each sentence into EDUs using a statistical model with lexical and syntactic features.
They evaluated the model by the ability to recover EDU boundaries and achieved the F1 of
83.1\% with automatically generated parsing tree and 84.7\% with golden truth parsing tree.
They also constructed a discourse parser and showed that near-human level performance can be achieved
if golden truth syntactic parsing tree and EDU segmentation were given. \cite{sporleder2005} focused
on non-hierarchical discourse chunking and produced comparable results without the use of
syntactic parsers. \cite{fisher2007utility} investigated whether the features derived from finite-state
systems are enough for discourse segmentation and showed that combining features from the syntactic tree
could improve the performance. \cite{joty2012novel} applied a CYK-like parsing algorithm
to sentence-level discourse parsing. By avoiding a greedy approach, they were able to increase
their performance.

Though sentence-level discourse parsing is well studied, document-level discourse
parsing still needs investigation. \cite{hernault2010hilda} attempted
document-level discourse parsing using Support Vector Machine (SVM) to
predict the discourse relations and construct the discourse tree in
a bottom-up fashion. \cite{feng2012text} further improved the results using more
linguistic features. \cite{joty2013combining} utilized CRFs to tackle this problem.
\cite{li2014recursive} experimented with a deep learning approach.
They recursively learn the representations for discourse units
while building the discourse tree in a bottom-up manner.


\section{Chinese Discourse Corpora}

There have been few Chinese discourse corpora compared to English until recently.
\cite{xue2005annotating} discussed the issues involved in annotating Chinese texts. They
pointed out the difficulty to determine argument spans due to the hierarchical
discourse structure. \cite{huang2011chinese} annotated 81 articles selected from
the Sinica Treebank 3.1~\footnote{http://turing.iis.sinica.edu.tw/treesearch/}. They followed
the top-level categories in PDTB for relation types and focused on the annotation
of inter-sentential discourse relations. \cite{huang2014interpretation} annotated the
relation types for 7,601 sentences that have only two discourse units.
\cite{zhou2012pdtb,zhou2015the} adopted PDTB-style discourse annotations and constructed
a Chinese discourse corpus that contains 164 articles selected from the \textit{Chinese
Treebank (CTB)}~\citep{xue2005penn}. \cite{zhang2014chinese} annotated 525 texts selected
from OntoNotes Release 4.0~\citep{weischedel_ontonotes_2011}. They have annotated
intra-sentential, inter-sentential, and passage-level relations, though
these annotations do not form a hierarchical structure naturally. \cite{zhou2014the}
refined the annotation scheme of PDTB2 and annotated 890 documents from CTB.
They focused mainly on intra-sentential discourse relations.

\cite{li2014building} proposed a \textit{Connective-driven Dependency Tree (CDT)}
scheme, which was designed specifically for Chinese rhetorical structure. They
annotated 500 documents from CTB. Each paragraph was segmented into EDUs similar
to the RST scheme. A three-level hierarchy of relation types was
proposed, and the explicit and implicit relations between different spans of EDUs
were annotated. The relations for each paragraph form a tree structure. In addition,
they annotated the discourse connectives for each relation. We use this corpus
for our study as it have both discourse connective annotations and discourse
structure annotations. We will discuss this corpus further in Section~\ref{s:CDTB}.

\section{Chinese Discourse Researches}


\cite{t1999applying,t2000enhancement} and \cite{chan2000mining} investigated
connective detection in Chinese texts as a part of a tagging system. \cite{hu2009research}
developed a fully automatic system to extract connective components from sentences.
They used a rule-based method and found that the performance was sensitive to
the connective lexicon. By removing words commonly used in non-discourse contexts,
they were able to improve accuracy from 58.66\% to 89.88\%.
\cite{zhou2012cross} and \cite{li2014cross} employed
cross-lingual information to deal with discourse usage ambiguity.
Their reported F scores are 79.33\% and 71.63\%, respectively.
\cite{li2014cross} used 5-way classification to classify a connective
between four relation types and non-discourse usage,
and obtained overall accuracy of 70.20\%. \cite{li2015automatic} used CDTB to
investigate detection and classification of connective components. They used
Maximum Entropy and decision tree algorithms with various syntactic features.
The F1 for connective identification was 69.2\% and the classification
accuracy among four top-level relation types was 89.1\% in the pipeline system.

\cite{hu2011research} dealt with linking ambiguity. They focused on sentences
that have multiple clauses and assumed all connective components in the sentence
are correctly identified. They developed a method to prune the search tree
to avoid the computation of all solutions while trying to find a connective
set that includes all connective components. They were able to achieve 98.9\%
accuracy in the 99 sentences that have linking ambiguities.

\cite{huang2011chinese} investigated relation type disambiguation between
two sentences. They used four-class classification with SVM, and their reported
accuracy and macro-averaged F1 are 88.28\% and 63.69\% respectively.
\cite{huang2012contingency} dealt with intra-sentential relations identification
and tackled the relation structure prediction with a 49-way classification
between different types of relation structures in a sentence.
