%
%   Chapter Introduction
%
%   Yong-Siang Shih
%   R.O.C.104.07
%
\chapter{Introduction}
\label{c:intro}

Natural language understanding is an ultimate goal in Artificial
Intelligence. There are many potential applications such as natural language
interface and large-scale content-analysis. However, NLP tasks are
challenging because there exist various kinds of ambiguities at different processing levels
including lexical analysis, syntactic analysis, semantic analysis, and discourse analysis.
Real world knowledge may be required to fully understand the concepts involved in the text.
As large-scale annotated datasets become available, researchers build upon each
component and start to handle the complex discourse relations between textual units.

In this thesis, we focus on several issues regarding Chinese discourse analysis
including the identification of explicit discourse connectives,
the identification of correct linking between the connective components,
the classification of relation types for connectives,
and the extraction of their arguments.

%
%   Background
%
\section{Background}

Discourse refers to a coherent sequence of sentences or phrases.
Discourse analysis attempts to extract meaningful information from
discourse units. Such discourse information includes exact boundaries
between discourse units and the discourse relation types between
discourse units.

Discourse relations represent how discourse units logically connect
with each other. There are different categorization systems of relation types.
For example, in
\textit{Penn Discourse Treebank 2.0 (PDTB2)}~\citep{Prasad08thepenn}, the discourse
relation types are defined in a three-level hierarchy. The top-level categories
include \textit{Temporal}, \textit{Contingency}, \textit{Comparison},
and \textit{Expansion}, while in
\textit{Chinese Discourse Treebank (CDTB)}~\citep{li2014building},
the top-level categories include \textit{Causality}, \textit{Coordination},
\textit{Transition}, and \textit{Explanation}.

In this thesis, we use CDTB as our corpus, so we will follow their taxonomy.
The meanings of these relation types are briefly explained as follows:

\begin{description}
\item[Causality] Causality is used when an event in an argument causes the event
    in another argument. It expresses the relationship between the cause
    and the effect.
\item[Coordination] Coordination is used when the arguments are
    descriptions on different aspects of the same thing or
    on different things that share common features.
\item[Transition] Transition is used when the arguments contrast with each other.
    It shows the difference between arguments.
\item[Explanation] Explanation expresses the same concept using different wordings.
    It is used for arguments that try to explain the same thing in different
    ways.
\end{description}


We will take some sentences from CDTB~\citep{li2014building} to illustrate the issues
in discourse analysis. The discourse relation types for the following examples
(S~\ref{sent:causality}-\ref{sent:explanation}) are
Causality, Coordination, Transition, and Explanation, respectively.
Each relation represents the relationship between two or more discourse units.
The discourse units that a relation involves are called its arguments.
For example, (S~\ref{sent:coordination}) has
three arguments, while (S~\ref{sent:transition}) has two arguments. The arguments are
denoted by [] markers in these examples.

% Causality
\begin{sent}{sent:causality}{}
    [上海近年来颁布了七十一件法规性文件,] [确保了开发的有序进行。]
    (Shanghai recently announced 71 regulatory documents, ensuring
    the order of development.)
\end{sent}

% Coordination
\begin{sent}{sent:coordination}{}
    [他们认为,到浦东新区投资办事有章法,] [讲规矩,] [利益能得到保障。]
    (Their view on investing in the Pudong New Area is that
    things are done methodologically,
    rules are followed, and their interest can be protected.)
\end{sent}

% Transition
\begin{sent}{sent:transition}{}
    [假新闻\underline{虽然}为数甚少,] [\underline{但}影响极坏。]
    (Although there is very little false news, it has a very bad influence.)
\end{sent}

% Explanation
\begin{sent}{sent:explanation}{}
    [他说,中国的农业生产将与人口同步增长,] [也就是说每年增长大约百分之一。]
    (He said that China's agricultural production will increase in line with
    population, that is, it will increase about one percent per year.)
\end{sent}

In addition to the relation types, each relation could also be classified
as implicit or explicit. An explicit relation is signaled by a discourse connective.
For example, in (S~\ref{sent:transition}), there is a connective ``虽然-但''
(although-but), so this is an explicit relation. On the other hand,
(S~\ref{sent:causality}) does not have a connective, so the relation
is represented implicitly.

These discourse connectives explicitly signal the presence of a discourse relation,
and therefore are important clues for discourse analysis.
In Chinese, there are many parallel connectives that have more than one
component~\citep{zhou2012pdtb}.
In this thesis, we call the components of a connective as its connective components.
Each connective can be composed of one or more components. For example, ``虽然-但''
(although-but) is composed of two connective components: ``虽然'' (although)
and ``但'' (but).

As discourse relations can form hierarchical structures in a text, sometimes
a connective may be present for a sub-structure, but the higher-level relation
is still implicit because the connective is not intended for that relation.
For example, (S~\ref{sent:hierarchy}) contains (S~\ref{sent:transition})
as a sub-structure. While (S~\ref{sent:transition}) is an explicit relation,
the discourse relation between the two discourse units in (S~\ref{sent:hierarchy})
is still implicit.

\begin{sent}{sent:hierarchy}{}
    [类似的假新闻和半真半假的新闻,并非《湖北日报》上有,其他报纸,以及电视、广播中也曾有过。]
    [假新闻虽然为数甚少,但影响极坏。]
    (Similar false news not only appears on Hubei Daily, but can also be found on
    other newspapers as well as TV and radio.
    Although there is very little false news, it has a very bad influence.)
\end{sent}



%
%   Motivation
%
\section{Motivation}

Researches for English discourse relations has been progressing continuously. This is
partly due to the availability of large-scale English discourse corpora including
\textit{Rhetorical Structure Theory Discourse Treebank (RST-DT)}~\citep{Carlson01building}
and PDTB2~\citep{Prasad08thepenn}. Comparatively, the available resources for Chinese discourse
researches lacked. Researchers often used self-constructed datasets to carry
out their experiments. Such datasets are relatively small and are often not publicly
available, making it difficult to compare the results from different researchers.

In recent years, several Chinese discourse corpora have become available. For example,
there are \textit{HIT-CDTB}~\citep{zhang2014chinese}, CDTB~\citep{li2014building}, and
\textit{Discourse Treebank for Chinese (DTBC)}~\citep{zhou2014the}. Therefore, it
becomes feasible to utilize these public datasets to comprehensively study
the issues in Chinese discourse analysis.

There are many unique challenges for Chinese language that are not present in
English discourse studies. Firstly, there are more discourse connectives
in Chinese than in English and their parts of speech have more
varieties~\citep{huang2014interpretation}. Therefore, it's easier to find words
that have the same surface forms as these connectives, but not actually have
discourse meaning. Thus, there is need to disambiguate between
discourse and non-discourse usages.
Secondly, many Chinese connectives are parallel connectives that
have multiple discontinuous components~\citep{zhou2012pdtb}.
When there are multiple connectives present in a paragraph, their components
can often link with each other in multiple possible ways.
Finding the correct linking between these components
can be useful for discourse analysis. Finally, the sentence boundaries for
Chinese texts are not clearly defined. Because there is no
requirement to always terminate a sentence under certain rules,
sometimes there could exist very long sentences. Thus it's more difficult
to detect discourse units and determine the arguments of a given relation.

(S~\ref{sent:invest}) shows an example for linking ambiguities.
There are five connective candidates, including (1) ``除了...还''
(in addition to ... also), (2) ``还...也'' (also ... also),
(3) ``除了'' (in addition to), (4) ``还'' (also), and (5) ``也'' (also).
Figure~\ref{i:linking-ambiguity} illustrates the ambiguous linking. Since each
component can only be a part of one connective, overlapped connectives could
not coexist.  In other words, only candidates (1) and (5) are correct.

\begin{sent}{sent:invest}{}
\underline{除了}投资环境优越,\underline{还}在于这些企业所具有的产品优势。
(...) 对上海的产业优化 \underline{也}有很大的带动作用。
(``\underline{In addition to}
superior investment environment, it's \underline{also} due to the product
advantages possessed by the enterprise. (...) It \underline{also} has
great effect in promoting the optimization of industries in Shanghai.'')
\end{sent}

%i:linking-ambiguity
\begin{figure}[!htbp]
\vspace{1em}
\centering
\tikzstyle{every picture}+=[remember picture]
\tikzstyle{na} = [shape=rectangle,inner sep=2pt]
\tikz\node[na](word1){\underline{除了}}; ... ,
\tikz\node[na](word2){\underline{还}}; ... 。 ... 
\tikz\node[na](word3){\underline{也}}; ... 。
\begin{tikzpicture}[overlay]
  \path[-, black,thick](word2) edge [out=90, in=90] (word1);
  \path[-, gray,thick](word2) edge [out=70, in=110] (word3);
\end{tikzpicture}

\caption{\label{i:linking-ambiguity} Ambiguous linking between connective components. }
\end{figure}


%
%   Goal
%
\section{Goal}

In this thesis, we aim at investigating these unique challenges for Chinese.
The goal of this research is to build an end-to-end system to analyze
the explicit discourse relations in Chinese texts. In particular, we
deal with four tasks. The first task is the extraction of explicit discourse
connectives. This includes the extraction of candidates and the disambiguation
between discourse and non-discourse usages. We will investigate the disambiguation
both on the connective component level and connective level.
To correctly identify parallel connectives, linking ambiguities between
the component candidates must be resolved. Therefore, our second task
focuses on resolving the linking ambiguities.
The third task is the classification of the relation type for each discourse
relation.
We will classify each relation represented by an explicit connective for top-level
relation types and second level relation types.
The fourth task is the extraction of the arguments for each discourse connective.
Built upon these components, an end-to-end discourse parser for Chinese may
be built in future studies.

%
%   Structure
%
\section{Structure}
This thesis is organized as follows. In Chapter \ref{c:related}, we summarize
the related work for discourse analysis, including the corpora and researches for
English and Chinese. In Chapter \ref{c:datasets}, we discuss the datasets used in
this research, and provide some statistics and analysis. In Chapter \ref{c:method},
we give an overview of our system, and discuss the design of various components.
In Chapter \ref{c:exp}, we perform evaluation on our models and compare with
other works. Finally, we conclude our work and discuss some future directions in
Chapter \ref{c:future}.
