%
%   Chapter Introduction
%
%   Yong-Siang Shih
%   R.O.C.104.07
%
\chapter{Introduction}
\label{c:intro}

Natural language understanding has always been an important goal in Artificial
Intelligence. There are many applications including natural language user
interface and large-scale content-analysis. However, the task is considered
as very difficult because there are often many ambiguities in a seemingly
simple sentence. Moreover, real world knowledge is sometimes required to fully
understand the concepts involved in the text. Researchers deal with this problem
by dividing it into subtasks such as lexical analysis, syntax analysis, semantic
analysis, and discourse analysis. As large-scale annotated datasets become
available, researchers build upon each component and start to handle the
complex discourse relations between textual units.

In this thesis, we focus on several issues regarding Chinese discourse analysis
including the identification and classification of explicit discourse connectives,
the identification of correct linkings between the connective components,
and the extraction of their arguments. 

%
%   Background
%
\section{Background}

Discourse is used to refer to a coherent sequence of sentences or phrases.
Discourse analysis is an attempt to extract meaningful information from
the discourse units. Such discourse information includes the exact boundaries
between the discourse units and the discourse relation types between
these discourse units.

Discourse relations represent how different discourse units logically connect
with each other. There are different categorization systems of relation types.
For example, in
\textit{Penn Discourse TreeBank 2.0 (PDTB2)}~\citep{Prasad08thepenn}, the discourse
relation types are defined in a three-level hierarchy. The top-level categories
include \textit{Temporal}, \textit{Contingency}, \textit{Comparison},
and \textit{Expansion}, while in
\textit{Chinese Discourse TreeBank (CDTB)}~\citep{li2014building}, 
the top-level categories include \textit{Causality}, \textit{Coordination},
\textit{Transition}, and \textit{Explanation}.

In this thesis, we use CDTB as our corpus, so we will follow their taxonomy.
The meanings of these relation types are briefly explained as follows:

\begin{description}
\item[Causality] Causality is used when an event in an argument cause the event
    in another argument. It expresses the relationship between the cause
    and the effect.
\item[Coordination] Coordination is used when the arguments are
    descriptions on different aspects of the same thing or
    on different things that share common features.
\item[Transition] Transition is used when the arguments contrast with each other.
    It shows the difference between arguments.
\item[Explanation] Explanation expresses the same concept using different wordings.
    It is used for arguments that try to explain the same thing in different
    ways.
\end{description}


We will take some sentences from CDTB~\citep{li2014building} to illustrate the issues
in discourse analysis. The discourse relation types for the following examples
(S~\ref{sent:causality}-\ref{sent:explanation}) are
Causality, Coordination, Transition, and Explanation, respectively.
Each relation represents the relationship between two or more discourse units.
The discourse units that a relation involves are called its arguments.
For example, (S~\ref{sent:coordination}) has
three arguments, while (S~\ref{sent:transition}) has two arguments. The arguments are
denoted by [] markers in these examples.

% Causality
\begin{sent}{sent:causality}{}
    [上海近年来颁布了七十一件法规性文件,] [确保了开发的有序进行。]
    (Shanghai recently announced 71 regulatory documents, ensuring
    the order of development.)
\end{sent}

% Coordination
\begin{sent}{sent:coordination}{}
    [他们认为,到浦东新区投资办事有章法,] [讲规矩,] [利益能得到保障。]
    (Their view on investing in the Pudong New Area is that
    things are done methodologically,
    rules are followed, and their interest can be protected.)
\end{sent}

% Transition
\begin{sent}{sent:transition}{}
    [假新闻\zhbf{虽然}为数甚少,] [\zhbf{但}影响极坏。]
    (Although there are very few false news, it has a very bad influence.)
\end{sent}

% Explanation
\begin{sent}{sent:explanation}{}
    [他说,中国的农业生产将与人口同步增长,] [也就是说每年增长大约百分之一。]
    (He said that China's agricultural production will increase in line with
    population, that is, it will increase about one percent per year.)
\end{sent}

In addition to these relation types, each relation could also be classified
as implicit or explicit. An explicit relation is signalled by a discourse connective.
For example, in (S~\ref{sent:transition}), there is a connective ``虽然-但''
(although-but), so this is an explicit relation. On the other hand,
(S~\ref{sent:causality}) does not have a connective, so the relation
is represented implicitly.

These discourse connectives explicitly signal the presence of a discourse relation,
and therefore are important clues for discourse analysis.
In Chinese, there are many parallel connectives that have more than one
component~\citep{zhou2012pdtb}.
In this thesis, we will denote the components of a connective as connective components.
Each connective can be composed of one or more components. For example, ``虽然-但''
(although-but) is composed of two connective components, ``虽然'' (although)
and ``但'' (but).

As discourse relations can form hierarchy structures in a text, sometimes
a connective may be present for a sub-structure, but the higher-level relation
is still implicit because the connective is not intended for that relation.
For example, (S~\ref{sent:hierarchy}) contains (S~\ref{sent:transition})
as a sub-structure. While (S~\ref{sent:transition}) is an explicit relation,
the discourse relation between the two discourse units in (S~\ref{sent:hierarchy})
is still implicit.

\begin{sent}{sent:hierarchy}{}
    [类似的假新闻和半真半假的新闻,并非《湖北日报》上有,其他报纸,以及电视、广播中也曾有过。]
    [假新闻虽然为数甚少,但影响极坏。]
    (Similar false news not only appears on Hubei Daily, but can also be found on
    other newspapers as well as TV and radio.
    Although there are very few false news, it has a very bad influence.)
\end{sent}



%
%   Motivation
%
\section{Motivation}

Research for English discourse relations has been progressing continuously. This is
partly due to the availability of large-scale English discourse corpora including
\textit{Rhetorical Structure Theory Discourse Treebank (RST-DT)}~\citep{Carlson01building}
and PDTB2~\citep{Prasad08thepenn}. Comparatively, the resource for Chinese discourse
research has been lacking. Researchers often used self-constructed datasets to carry
out their experiments. Such datasets are relatively small and are often not publicly
available, making it difficult to compare the results from different researchers.

In recent years, several Chinese discourse corpora have become available. For example,
there are \textit{HIT-CDTB}~\citep{zhang2014chinese}, CDTB~\citep{li2014building}, and
\textit{Discourse Treebank for Chinese (DTBC)}~\citep{zhou2014the}. Therefore, it
becomes feasible to utilize these public datasets to comprehensively study
the issues in Chinese discourse analysis.

There are many unique challenges for Chinese language that are not present in
English discourse studies. Firstly, there are more discourse connectives
in Chinese than in English and their parts of speech have more
varieties~\citep{huang2014interpretation}. Therefore, it's easier to find words
that have the same surface forms as these connectives, but not actually have
discourse meaning. Therefore, there is need to disambiguate between
discourse and non-discourse usages.
Secondly, many Chinese connectives are parallel connectives that
have multiple discontinuous components~\citep{zhou2012pdtb}.
When there are multiple connectives present in a paragraph, their components
can often link with each other in multiple possible ways.
Finding the correct linkings between these components
can be useful for discourse analysis. Finally, the sentence boundaries for
Chinese texts are not clearly defined. Because there is no
requirement to always terminate a sentence under certain rules,
sometimes there could exist very long sentences. Thus it's more difficult
to detect discourse units and determine which are the arguments of a given relation.

%
%   Goal
%
\section{Goal}

In this thesis, we try to investigate these unique challenges for Chinese.
The goal of this research is to build an end-to-end system to analyze
the explicit discourse relations in Chinese texts. In particular, we
deal with four tasks. The first task is the extraction of explicit discourse
connectives. This includes the extraction of candidates and the disambiguation
between discourse and non-discourse usages. We will firstly build a basic model
to identify connective components in the text, and afterwards, we will
integrate linking relationships between the components to improve the performance.
The second task is the disambiguation for linkings between connective components.
We identify how the connective components pair with each others to form connectives.
The third task is the classification of the relation type for each discourse relation.
We will classify each relation represented by a explicit connective by top-level
relation types and second level relation types.
The fourth task is the extraction of the arguments for each discourse connective.
By dealing with these issues, an end-to-end discourse parser for Chinese may
be built in future studies.

%
%   Structure
%
\section{Structure}
This thesis is organized as follows. In Chapter \ref{c:related}, we summarize
the related work for discourse analysis, including the corpora and research for
English and Chinese. In Chapter \ref{c:corpus}, we discuss the data sets used in
this research, and provide some statistics and analysis. In Chapter \ref{c:method},
we give an overview of our system, and discuss the design of various components.
In Chapter \ref{c:exp}, we perform evaluation on our models and compare with
other works. Finally, we conclude our work and discuss some future directions in
Chapter \ref{c:future}.
