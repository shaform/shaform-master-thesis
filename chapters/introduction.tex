%
%   Chapter Introduction
%
%   Yong-Siang Shih
%   R.O.C.104.07
%
\chapter{Introduction}
\label{c:intro}

Natural language understanding has always been an important goal in Artificial Intelligence.
There are many applications including natural language user interface and large-scale content-analysis.
However, the task is also considered as very difficult because there are often many ambiguities in a seemingly
simple sentence. Moreover, real world knowledge is required to fully understand the concepts
involved in the text.

Many subtasks, one of them is discourse analysis.

%
%   Background
%
\section{Background}

Talk about discourse research.




%
%   Motivation
%
\section{Motivation}

Talk about Chinese. and our goal.

There are many challenges unique to Chinese.

The goal of this research is to build an end-to-end system to analyze the explicit discourse
connectives in Chinese texts. This includes (1) the extraction of discourse connective
components, (2) the disambiguation for linkings between connective components, (3) the classification
of the relation type for each discourse, and (4) the extraction of arguments for each discourse connective.

Such a system could be incorporated. Eventually build a discourse parser.

%
%   Structure
%
\section{Structure}
This thesis is organized as follows.
