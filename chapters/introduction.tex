%
%   Chapter Introduction
%
%   Yong-Siang Shih
%   R.O.C.104.07
%
\chapter{Introduction}
\label{c:intro}

Natural language understanding has always been an important goal in Artificial
Intelligence. There are many applications including natural language user
interface and large-scale content-analysis. However, the task is considered
as very difficult because there are often many ambiguities in a seemingly
simple sentence. Moreover, real world knowledge is sometimes required to fully
understand the concepts involved in the text. Researchers deal with this problem
by dividing it into subtasks such as lexical analysis, syntax analysis, semantic
analysis, and discourse analysis. As large-scale annotated data sets become
available, researchers build upon each component and start to handle the
complex discourse relations between textual units.

In this thesis, we focus on several issues regarding Chinese discourse analysis
including the identification and classification of explicit discourse connectives,
the identification of correct linkings between the connective components,
and the extraction of their arguments. 

%
%   Background
%
\section{Background}

Discourse is used to refer to a coherent sequence of sentences or phrases.
Discourse analysis is an attempt to extract meaningful information from
the discourse units. Such discourse information includes the exact boundaries
between the discourse units and the discourse relation types between
these discourse units.

Discourse relations represent how different discourse units logically connect
with each other.  There are different categorization systems of relation types, but
it's usually defined in a hierarchical manner so that there are different number
of relation types depending on granularity. For example, in
\textit{Chinese Discourse TreeBank (CDTB)}~\citep{li2014building}, the discourse
relation types are defined in a three-level hierarchy. The top-level categories
include \textit{Causality}, \textit{Coordination}, \textit{Transition},
and \textit{Explanation},
while in \textit{Penn Discourse TreeBank (PDTB) 2.0}~\citep{Prasad08thepenn},
the top-level categories include \textit{Temporal}, \textit{Contingency},
\textit{Comparison}, and \textit{Expansion}. In this thesis, we use
CDTB as our corpus, so we will follow their taxonomy. The meanings of these
relation types are briefly explained as follows:

\begin{description}
\item[Causality] Causality is used when an event in an argument cause the event
    in another argument. It expresses the relationship between the cause
    and the effect.
\item[Coordination] Coordination is used when the arguments are
    descriptions on different aspects of the same thing or
    on different things that share common features.
\item[Transition] Transition is used when the arguments contrast with each other.
    It shows the difference between arguments.
\item[Explanation] Explanation expresses the same concept using different wordings.
    It is used for arguments that try to explain the same thing in different
    ways.
\end{description}

The following are example sentences for each relation type respectively.

% Causality
\begin{sent}{sent:causality}{}
    [上海近年来颁布了七十一件法规性文件,] [确保了开发的有序进行。]
    (Shanghai recently announced 71 regulatory documents, ensuring
    the order of development.)
\end{sent}

% Coordination
\begin{sent}{sent:coordination}{}
    [他们认为,到浦东新区投资办事有章法,] [讲规矩,] [利益能得到保障。]
    (Their view on investing in the Pudong New Area is that
    things are done methodologically,
    rules are followed, and their interest can be protected.)
\end{sent}

% Transition
\begin{sent}{sent:transition}{}
    [假新闻\textbf{虽然}为数甚少,] [\textbf{但}影响极坏。]
    (Although there are very few false news, it has a very bad influence.)
\end{sent}

% Explanation
\begin{sent}{sent:explanation}{}
    [他说,中国的农业生产将与人口同步增长,] [也就是说每年增长大约百分之一。]
    (He said that China's agricultural production will increase in line with
    population, that is, it will increase about one percent per year.)
\end{sent}

In addition to these relation types, each relation could also be classified
as implicit or explicit. An explicit relation is signalled by a discourse connective.
For example, in (S~\ref{sent:transition}), there is a connective ``虽然-但''
(although-but), so this is an explicit relation. On the other hand,
(S~\ref{sent:causality}) does not have a connective, so the relation
is represented implicitly.

In Chinese, there are many parallel connectives that have more than one components.
In this thesis, we will denote the components of a connective as connective components.
Each connective can be composed of one or more components. For example, ``虽然-但''
(although-but) is composed of two connective components, ``虽然'' (although)
and ``但'' (but).

As discourse relations can have hierarchy structures in a text, sometimes
a connective may be present for a sub-structure, but the higher-level relation
is still implicit because the connective is not intended for that relation.
For example, (S~\ref{sent:hierarchy}) contains (S~\ref{sent:transition})
as a sub-structure. While (S~\ref{sent:transition}) is an explicit relation,
(S~\ref{sent:hierarchy}) is still an implicit relation.

\begin{sent}{sent:hierarchy}{}
    [类似的假新闻和半真半假的新闻,并非《湖北日报》上有,其他报纸,以及电视、广播中也曾有过。]
    [假新闻虽然为数甚少,但影响极坏。]
    (Similar false news not only appears on Hubei Daily, but can also be found on
    other newspapers as well as TV and radio.
    Although there are very few false news, it has a very bad influence.)
\end{sent}

Finally, the discourse units that a relation involves are called its arguments.
Each relation has two or more arguments. For example, (S~\ref{sent:coordination}) has
three arguments, while (S~\ref{sent:transition}) has two arguments. The arguments are
denoted by [] markers in these examples.


%
%   Motivation
%
\section{Motivation}

Talk about Chinese. and our goal.

There are many challenges unique to Chinese language. For example, many of the connectives
in Chinese are paired connectives that are composed of multiple components.

Because sentences do not have a clear boundaries in Chinese texts. It could easily range
over extremely long text.

paired connectives a lot here \citep{zhou2012pdtb}


%
%   Goal
%
\section{Goal}

The goal of this research is to build an end-to-end system to analyze the explicit discourse
connectives in Chinese texts. The first task is the extraction of explicit discourse
connectives. This includes the extraction of candidates and the disambiguation between discourse
and non-discourse usages. We will firstly build a basic model to identify connective components
in the text, and then integrate linking relationships between the components to improve
the performance. The second task is the disambiguation for linkings between connective components.
We identify how the connective components pair with each others to form connectives.
The third task is the classification of the relation type for each discourse relation. We classify
each relation represented by a explicit connective.
The fourth component is the extraction of arguments of each discourse connective.

%
%   Structure
%
\section{Structure}
This thesis is organized as follows.
