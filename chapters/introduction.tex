%
%   Chapter Introduction
%
%   Yong-Siang Shih
%   R.O.C.104.07
%
\chapter{Introduction}
\label{c:intro}

Natural language understanding has always been an important goal in Artificial
Intelligence. There are many applications including natural language user
interface and large-scale content-analysis. However, the task is also considered
as very difficult because there are often many ambiguities in a seemingly
simple sentence. Moreover, real world knowledge is sometimes required to fully
understand the concepts involved in the text. Researchers deal with this problem by dividing it into subtasks such as lexical
analysis, syntax analysis, semantic analysis, and discourse analysis.
As large-scale annotated data sets become available, researchers build
upon each component that was perfected over the years and start to handle the
complex discourse relations between textual units.

In this thesis, we focus on several issues regarding Chinese discourse analysis
including the identification and classification of explicit discourse connectives,
the identification of correct linkings between these connectives, and the extraction
of their arguments. 

%
%   Background
%
\section{Background}

Discourse is used to refer to a coherent sequence of sentences or phrases.
Discourse analysis is an attempt to extract meaningful information from
these sentences. Such discourse information includes the exact boundaries
between the discourse units and the discourse relation types between
these discourse units.

Discourse relations represent how different discourse units logically connect
with each other.  There are different categorization systems of relation types, but
it's usually defined in a hierarchical manner so there are different number
of relation types depending on granularity. For example, in
\textit{Chinese Discourse TreeBank (CDTB)} \citep{li2014building}, the discourse
relation type are defined in a three-level hierarchy. The top-level categories
include \textit{Causality}, \textit{Transition}, \textit{Coordination}
and \textit{Explanation}. In PDTBwhere the four top-level . We briefly explain their meanings as follows:

\begin{description}
\item[Causality] Some explanations \ldots
\item[Transition] Some explanations \ldots
\item[Coordination] A Coordination relation is used when the arguments are
    descriptions on different aspects of the same thing or
    on different things that share common features.
\item[Explanation] Some explanations \ldots
\end{description}

Some example sentences \ldots

Some explanation about explicit and implicit relations \ldots



%
%   Motivation
%
\section{Motivation}

Talk about Chinese. and our goal.

There are many challenges unique to Chinese.


%
%   Goal
%
\section{Goal}

The goal of this research is to build an end-to-end system to analyze the explicit discourse
connectives in Chinese texts. The first task is the extraction of explicit discourse
connectives. This includes the extraction of candidates and the disambiguation between discourse
and non-discourse usages. We will firstly build a basic model to identify connective components
in the text, and then integrate linking relationships between the components to improve
the performance. The second task is the disambiguation for linkings between connective components.
We identify how the connective components pair with each others to form connectives.
The third task is the classification of the relation type for each discourse relation. We classify
each relation represented by a explicit connective.
The fourth component is the extraction of arguments of each discourse connective.

%
%   Structure
%
\section{Structure}
This thesis is organized as follows.
