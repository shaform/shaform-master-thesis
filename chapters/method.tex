%
%   Chapter Methods
%
%   Yong-Siang Shih
%   R.O.C.104.07
%
\chapter{Methods}
\label{c:method}

\section{Connective Candidate Extraction}

To identify the correct discourse connectives, we need to
extract connective candidates and distinguish between discourse
and non-discourse usages among these candidates.

Firstly, we attempt the detection of connective component candidates.
The simplest method is to just use string matching with the connective
component lexicon we obtained to extract all possible instances. This yields
24,539 candidates, while only 2,131 of them are correct instances. An improvement
could be made by making the observation that many of the components only have
discourse meaning when they are paired with other components. So instead of matching
with the connective component lexicon, we use connective lexicon and only extract
a component candidate when it forms a single connective or pairs with
other candidates to form a paired connective. This leaves us with 12,526 candidates.

One reason for many incorrect instances is that many characters used for connectives
appear in other unrelated words. For example, even though ``如'' (if) is a connective,
this character appears in an unrelated word ``如是说'' (says). To alleviate this problem,
Stanford Chinese segmenter~\citep{chang2008optimizing} is employed to segment paragraphs
into tokens. These token boundaries can be used as clues for eliminating spurious candidates.
In particular, we only extract a component when it can be composed by complete tokens.
For example, in (S~\ref{sent:keyissue}) we extract ``不是...而是'' (not ... but) as a candidate
even though ``不'' (not) and ``是'' (is) are separated. Comparatively, in (S~\ref{sent:improve}),
the correct connective ``如'' (if) is not extracted because it does not satisfy a token boundary.

\begin{sent}{sent:keyissue}{}
    当前 / 经济 / 的 / 关键 / \underline{不} / \underline{是} / 争取 / 更 /
    高 / 的 / 增长 / 速度 / , / \underline{而是} / 提高 / 效益 / 。
    (The key issue for current economy is not to pursue faster growth, but to
    increase productivity.)
\end{sent}

\begin{sent}{sent:improve}{}
    \underline{如}无 / 改进 / , / 很 / 难 / 在 / 海外 / 市场 / 参与 / 竞争 /
    。 (It will be difficult to compete in foreign markets if there is no
    improvement.)
\end{sent}

Using this procedure, only 7,649 component candidates are extracted, with 2,068 of 2,131 annotated
components recovered. These candidates could form 7,976 connective candidates, recovering
1,755 of 1,813 annotated connectives. Thus, 0.9704 and 0.9680 become upper bounds of recall
for the subsequent stages. Table~\ref{t:cand-extract} shows the comparison between different
extraction methods. The technique we use reduce spurious candidates substantially while
maintaining high recall.

%t:cand-extract
\begin{table}[ht]
\centering
\begin{tabular}{|c|c|c|c|}
\hline
Method                  & component lexicon & connective lexicon & +segmentation \\ \hline

Component Recall        & 1                 & 1                  & 0.9704        \\ \hline
Component Precision     & 0.0868            & 0.1705             & 0.2704        \\

\hhline{|=|=|=|=|}

Connective Recall       & 1                 & 1                  & 0.9680        \\ \hline
Connective Precision    & 0.1196            & 0.1196             & 0.2200        \\ \hline

\end{tabular}
\caption{\label{t:cand-extract} Comparison between different candidate extraction methods.}
\end{table}

