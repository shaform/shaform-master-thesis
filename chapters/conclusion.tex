%
%   Chapter Conclusion
%
%   Yong-Siang Shih
%   R.O.C.104.07
%
\chapter{Conclusion and Future Work}
\label{c:future}

\section{Conclusion}

In this thesis, we investigate several issues
regarding Chinese discourse analysis. Our
research focuses on the explicit relations that
are signaled by discourse connectives.
We tackle the tasks including discourse
usage disambiguation of connectives,
linking disambiguation among connective components,
relation type disambiguation, and argument
extraction for explicit relations. A pipeline
system is built to identify explicit relations
in Chinese text.

We propose several features
for discourse usage disambiguation.
We have found that word embeddings are useful in this
task. The same scores predicted by Logistic Regression
are also used to rank the candidates, and 
a greedy algorithm is developed to resolve linking
ambiguities. We find that one of the challenging issues
is that many spurious connective candidates are formed by 
spurious component candidates. Our model achieves
an F1 of 78.81\% for extracting connective components,
and an F1 of 74.97\% for extracting connectives. Similar
features are used to classify the relation type for each
connective instance, and we achieve macro-averaged F1 scores
95.07\% and 76.92\% among top-level relation types and
second-level relation types, respectively.

Additionally, we investigate argument extraction for
explicit relations. Each paragraph is divided into segments
and the task is formulated as a sequence labeling problem
over the segments.
We achieve an F1 of 78.48\% for identifying
argument boundaries, which outperforms the baseline system
significantly. However, correctly identifying all arguments
for an explicit discourse relation remains a difficult task. Our
system only achieves an accuracy of 40.74\%. We show
the challenging issues include the determination of
the interval the arguments span. Due to the hierarchical
nature of the discourse structure, the ranges for
different relations can vary greatly.

The system components are integrated into a pipeline
system that extracts explicit discourse relations from the raw text.
It extracts all connectives, determines their relation types,
and identifies their arguments.
The pipeline system achieves an overall F1 of 31.31\% under
the same dataset.

\section{Future Work}

There still exist some issues that need to be further investigated
before a Chinese discourse parser could be built. In this section,
we discuss some improvement that could be made to our models and
future work that could be built upon our work.

\subsection{Integration of Discourse Usage Disambiguation and Linking Disambiguation}

In our work, a two-stage pipeline system is built to disambiguate
the discourse usages and the linking ambiguities. Although some linking
information is used as features, the greedy algorithm still ranks
each connective candidate individually. We expect that a closer integration
of these two stages and an algorithm that utilizes global relationship
between different conflicting candidates may improve the performance
for both tasks.

\subsection{Utilizing Connective Arguments for Discourse Relation Disambiguation
and Discourse Usage Disambiguation}

Although we did not find the relation type useful for
argument extraction, the argument spans could be useful for determining
the relation types.  Additional features that capture the relationship between
different arguments may be utilized to disambiguate between different relation
types. Likewise, the results for argument extraction might help
discourse usage disambiguation. For example, being unable to extract the arguments for
a particular connective candidate might suggest that the connective candidate
being considered might not be an actual connective in the first place.
However, the accuracy for argument identification must be improved
substantially to make it feasible to utilize the arguments extracted.

\subsection{Identifying the Implicit Relations for Discourse Parsing}

One challenging issue we encounter is the difficulty to determine
the exact argument interval for a given connective. Our model extracts
the arguments for each connective individually. Further investigation
is needed to use the hierarchical relationship between different
relations to improve the performance.

However, implicit relations must also be identified in order
to construct the complete discourse tree for a paragraph.
Investigation must be done to examine how to integrate the implicit and
explicit relation identification together, and whether our model
could help construct the hierarchical discourse structure.

Also, a connective's argument interval is constrained by the
argument boundaries of its parent node in the discourse tree. As
the parent node could be an implicit relation, taking
both implicit and explicit relations into consideration may also help
the determination of the argument interval for each connective.
